\documentclass[12pt]{article}
%\documentclass[double]{ua-thesis}
%\documentclass{ua-thesis}
\usepackage{amssymb,amsmath,graphicx,amsthm}
\usepackage[toc,page]{appendix}
% ----------------------------------------------------------------
% ----------------------------------------------------------------
%Page setup
%\oddsidemargin=0in \evensidemargin=0in \textwidth=6.3in \topmargin=-0.5in \textheight=9in
% ----------------------------------------------------------------
%\vfuzz2pt % Don't report over-full v-boxes if over-edge is small
%\hfuzz2pt % Don't report over-full h-boxes if over-edge is small
\newcommand{\abs}[1]{\left\vert#1\right\vert}
\newcommand{\st}{\ensuremath{\,|\,}}
\newcommand{\pr}{\ensuremath{^{\prime}}}
\newcommand{\C}{\ensuremath{\mathbb{C}}}
\newcommand{\Chat}{\ensuremath{\hat{\mathbb{C}}}}
\newcommand{\R}{\ensuremath{\mathbb{R}}}
\newcommand{\di}[1]{\ensuremath{\frac{\partial}{\partial #1}}}
\newcommand{\I}{\ensuremath{\mathbf{i}}}
\newcommand{\dlim}{\displaystyle\lim}
\newcommand{\dint}{\displaystyle\int}
\newcommand{\frag}{\ensuremath{\mathfrak{g}}}
\newcommand{\fragR}{\ensuremath{\mathfrak{g}^{\R{}}}}
\newcommand{\frau}{\ensuremath{\mathfrak{u}}}
\newcommand{\frah}{\ensuremath{\mathfrak{h}}}
\newcommand{\frakk}{\ensuremath{\mathfrak{k}}}
\newcommand{\frap}{\ensuremath{\mathfrak{p}}}
\newcommand{\fraa}{\ensuremath{\mathfrak{a}}}
\newcommand{\fran}{\ensuremath{\mathfrak{n}}}
\newcommand{\fraq}{\ensuremath{\mathfrak{q}}}
\newcommand{\fraksu}{\ensuremath{\mathfrak{su}}}
\newcommand{\fraksl}{\ensuremath{\mathfrak{sl}}}
\newcommand{\fral}{\ensuremath{\mathfrak{l}}}
\newcommand{\frat}{\ensuremath{\mathfrak{t}}}
\newcommand{\Cinfty}{\ensuremath{C^{\infty}}}
\newcommand{\dd}[1]{\ensuremath{\frac{\partial}{\partial #1}}}
\newcommand{\CalO}{\ensuremath{\mathcal{O}}}
\newcommand{\CalD}{\ensuremath{\mathcal{D}}}
\newcommand{\CalC}{\ensuremath{\mathcal{C}}}
\newcommand{\CalM}{\ensuremath{\mathcal{M}}}
\newcommand{\CalL}{\ensuremath{\mathcal{L}}}
\newcommand{\CalH}{\ensuremath{\mathcal{H}}}


\newcommand{\res}{\text{res}}
\newcommand{\Inv}{\mathrm{Inv}}
\newcommand{\Ind}{\mathrm{Ind}}
\newcommand{\Stab}{\mathrm{Stab}}
\newcommand{\ad}{\mathrm{ad}}
\newcommand{\Ad}{\mathrm{Ad}}
\newcommand{\Sym}{\ensuremath{\mathrm{Sym}}}
\newcommand{\Span}{\mathrm{span}}
\newcommand{\tr}{\mathrm{tr}}
%\newcommand{\dim}{\text{dim}}
\newcommand{\pd}[2]{\ensuremath{\frac{\partial #1}{\partial #2}}}

%--------------------------------------------------------------------------------------------------

%Theorems, Lemmas, etc.
\theoremstyle{plain}
\newtheorem{thm}{Theorem}[section]
\newtheorem{lemma}[thm]{Lemma}
\newtheorem{prop}[thm]{Proposition}
\newtheorem{cor}[thm]{Corollary}

\theoremstyle{definition}
\newtheorem{defn}[thm]{Definition}
\newtheorem{conj}[thm]{Conjecture}
\newtheorem{example}[thm]{Example}
\newtheorem{rmk}[thm]{Remark}
% ----------------------------------------------------------------
\begin{document}
\begin{center}
{\large Practice Project - Word Jumble} \linebreak Summer Research 2017
\end{center}

\begin{enumerate}

\item Write a Python program that solves the word jumble: The user enters a string of letters, and the program lists all possible permutations of those letters that form English words.  Use the provided dictionary (dictionary.txt).  Comment your code clearly.    

You may assume that the user's input consists only of letters, but you should convert all of those letters to uppercase.

Tools that will be needed:
\begin{enumerate}
	\item the \verb+input()+ function
	\item lists
	\item strings
	\item functions
	\item the \verb+range()+ function
	\item loops
	\item conditionals
	\item reading from and (perhaps) writing to files
\end{enumerate}

Your program should run more or less instantaneously.  If it doesn't, then you need to construct a more efficient algorithm.  

\textbf{Hint:}  It may be convenient to reorganize the information in the file dictionary.txt.  Feel free to create new files.

\item Find the string of letters such that permutations of that string form the maximum number of distinct English words.  For example, exactly two of the permutations of the string 'EGHHIT' are English words: EIGHTH and HEIGHT.  (This is \textbf{not} the maximum.) 

\textbf{Big Hint:}  Use the \verb+Counter()+ class from the \verb+collections+ module.


\end{enumerate}

\end{document}
% ----------------------------------------------------------------
